\documentclass{article}
\usepackage{tikz}
\title{Algoritmi v bioinformatiki - 2. Domača naloga}
\author{Jan Panjan}
\date{\today}

\begin{document}

\maketitle
% \newpage

\begin{enumerate}
	\item  \textit{Dano imamo naslednje zaporedje izidov metov kovanca}

		\begin{center}
			$V = CCCGCGGCCGC$
		\end{center}

		\textit{pri čemer $C$ označuje, da je bil izid meta cifra, $G$ pa da je bil izid meta grb.
		Za mete imamo na voljo 3 kovance, $A$, $B$ in $C$, veljajo naslednje verjetnosti:}

		\textit{Prehod:}

		\begin{center}
			\begin{tabular}{c||c|c|c}
				\% & A & B & C \\
				\hline
				\hline
				A & 40 & 30 & 30 \\
				\hline
				B & 30 & 40 & 30 \\
				\hline
				C & 30 & 30 & 40 \\
			\end{tabular}
		\end{center}

		\textit{Izpis:}

		\begin{center}
			\begin{tabular}{c||c|c}
				\% & C & G \\
				\hline
				\hline
				A & 75 & 25 \\
				\hline
				B & 80 & 20 \\
				\hline
				C & 20 & 80 \\
			\end{tabular}
		\end{center}

		\textit{Katera od možnosti je najbolj verjetna?}

		\begin{enumerate}
			\item \textit{za vse mete smo uporabili kovanec A}
			\item \textit{za vse mete smo uporabili kovanec C}
			\item \textit{za vse mete smo uporabili kovanec B}
			\item \textit{$\Pi$ = AAACBCCBBCA}
		\end{enumerate}

		\textit{Odgovor ustrezno utemeljite.}

		\textbf{Za vse mete smo uporabili kovanec A}

		\begin{tabular}{c||l}
			$(0.75)^7$ & 7-krat vržemo C z verjetnostjo $0.75$ \\
			\hline
			$(0.25)^2$ & 7-krat vržemo G z verjetnostjo $0.75$ \\
			\hline
			$(0.4)^{10}$ & 10-krat ne zamenjamo kovanca $A$ z verjetnostjo $0.4$
		\end{tabular}

		\begin{center}
			$p(A) = (0.75)^7 \cdot (0.25)^4 \cdot (0.4)^{10} = 0.00209$
		\end{center}

		\textbf{Za vse mete smo uporabili kovanec B}

		\begin{tabular}{c||l}
			$(0.8)^7$ & 7-krat vržemo C z verjetnostjo $0.8$ \\
			\hline
			$(0.2)^2$ & 7-krat vržemo G z verjetnostjo $0.2$ \\
			\hline
			$(0.4)^{10}$ & 10-krat ne zamenjamo kovanca $B$ z verjetnostjo $0.4$
		\end{tabular}

		\begin{center}
			$p(B) = (0.8)^7 \cdot (0.2)^4 \cdot (0.4)^{10} = 0.00134$
		\end{center}

		\textbf{Za vse mete smo uporabili kovanec C}

		\begin{tabular}{c||l}
			$(0.2)^7$ & 7-krat vržemo C z verjetnostjo $0.8$ \\
			\hline
			$(0.8)^2$ & 7-krat vržemo G z verjetnostjo $0.2$ \\
			\hline
			$(0.4)^{10}$ & 10-krat ne zamenjamo kovanca $C$ z verjetnostjo $0.4$
		\end{tabular}

		\begin{center}
			$p(C) = (0.2)^7 \cdot (0.8)^4 \cdot (0.4)^{10} = 0.0000209$
		\end{center}

		\textbf{$\Pi$ = AAACBCCBBCA}

		\begin{tabular}{c||l}
			$(0.3)^6$ & 6-krat ostanemo v istem kovancu (vsi kovanci imajo enake verjetnosti) \\
			\hline
			$(0.4)^4$ & 4-krat zamenjamo kovanec (tudi tu imajo enako verjetnosti) \\
			\hline
			$(0.75)^4$ & 4-krat vržemo kovanec $A$, vsakič vržemo cifro z verjetnostjo $0.75$ \\
			\hline
			$(0.8)^3$ & 3-krat vržemo kovanec $B$, vsakič vržemo cifro z verjetnostjo $0.8$ \\
			\hline
			$(0.8)^4$ & 4-krat vržemo kovanec $C$, vsakič vržemo grb z verjetnostjo $0.8$
		\end{tabular}

		\begin{center}
			$p(\Pi) = (0.3)^6 \cdot (0.4)^4 \cdot (0.75)^4 \cdot (0.8)^3 \cdot (0.8)^4 = 0.00000124$
		\end{center}

		\textbf{Rešitev:} Najbolj verjetna je možnost z največjo verjetnostjo. To je možnost (a)
		z verjetnostjo $0.00209$.

	\item \textit{Dani imamo zaporedji $s=GAGTACA$ in $t=TGATTACA$ ter vrednostno funkcijo s parametroma
		$\mu = 4, \sigma = 2$ in nagrado za ujemanje 2.}

		\begin{enumerate}
			\item \textit{Z uporabo Needleman-Wunsch-evega algoritma za globalno poravnavo smo dobili
				naslednjo tabelo:}

				 -- slika tabele --

				\textit{Dopolnite tabelo tako, da poračunate vrednosti (in ustrezne puščice) za zadnji dve vrstici.}

			\item \textit{Koliko optimalnih globalnih poravnav dobite? Izpišite vse rešitve.}
		\end{enumerate}

	\item \textit{Dano imamo naslednjo matriko izražanja:}

		\begin{center}
			\begin{tabular}{c||c|c|c|c|c|c|}
				& $T_1$ & $T_2$ & $T_3$ & $T_4$ & $T_5$ & $T_6$ \\
				\hline
				\hline
				$g_1$ & 2 & 2 & 6 & 2 & 3 & 4 \\
				\hline
				$g_2$ & 3 & 7 & 3 & 1 & 9 & 3 \\
				\hline
				$g_3$ & 2 & 2 & 7 & 2 & 6 & 3 \\
				\hline
				$g_4$ & 3 & 2 & 3 & 2 & 1 & 3 \\
				\hline
				$g_5$ & 2 & 1 & 5 & 1 & 0 & 4 \\
				\hline
				$g_6$ & 3 & 5 & 5 & 8 & 2 & 3 \\
				\hline
				$g_7$ & 1 & 3 & 1 & 5 & 4 & 2 \\
				\hline
				$g_8$ & 5 & 4 & 2 & 4 & 7 & 5
			\end{tabular}
		\end{center}

		\item \textit{Določite gruče z uporabo metode voditeljev, če je začetna množica voditeljev
			enaka $X = \{g_1, g_5, g_6\}$.}

		\item \textit{Izračunajte drevo hierarhičnega gručenja z uporabo algoritma UPGMA.}

\end{enumerate}
\end{document}
